\documentclass[10pt,a4paper]{article}
\usepackage{ngerman}
\usepackage[utf8]{inputenc}
\usepackage{sectsty, xcolor}
\usepackage{lastpage}
\usepackage{amssymb, enumitem, fancyhdr, graphicx, float, makeidx, textcomp, multicol}
\usepackage[hidelinks]{hyperref}
\usepackage[hang,flushmargin]{footmisc}
\makeindex

\definecolor{dunkelblau}{rgb}{0,0.4,0.6}
\subsectionfont{\color{dunkelblau}}

\title{WEBT FS 2020}
\author{Victor Fernández}
\date{Februar 2020}

\addtolength{\oddsidemargin}{-.875in}
\addtolength{\evensidemargin}{-.875in}
\addtolength{\textwidth}{1.75in}
\addtolength{\topmargin}{-.875in}
\addtolength{\textheight}{1.75in}

% muss nach Änderung der margin kommen!
\pagestyle{fancy}
\fancyhf{} %reset
\fancyhead[L]{HSLU}
\fancyhead[C]{WEBT}
\fancyhead[R]{\thepage/\pageref{LastPage}}
\fancyfoot[L]{}
\fancyfoot[C]{}
\fancyfoot[R]{}
\renewcommand{\headrulewidth}{0.2pt} % Strich in Kopfzeile

\begin{document}

\maketitle
\tableofcontents
\thispagestyle{empty}
\pagebreak

\part{SW01 - Einführung}
\section{Inhalt}
\begin{itemize}[noitemsep,topsep=0pt,leftmargin=*]
    \item Klassifikation von Web-Applikationen
    \begin{itemize}[noitemsep,topsep=0pt,leftmargin=*]
        \item Statische Webseiten
        \item Dynamische Webseiten
        \begin{itemize}[noitemsep,topsep=0pt,leftmargin=*]
            \item Serverseitige Erzeugung: Datenbank-orientiert
            \item Serverseitige Erzeugung: mit Webservices
            \item Clientseitige Erzeugung
        \end{itemize}
    \end{itemize}
    \item Formen mobiler Applikationen
    \item Grundlegende Adressierungskonzepte
\end{itemize}

\subsection{Verstehen der verschiedenen Varianten von Web Applikationen}
\paragraph*{Statische Websites}Einfache statische Webauftritte, z.B. Homepage
\begin{itemize}[noitemsep,topsep=0pt,leftmargin=*]
    \item Vorteile
    \begin{itemize}[noitemsep,topsep=0pt,leftmargin=*]
        \item Einfache Basistechnologien: HTML und CSS
        \item Einfache Infrastruktur notwendig: Webserver
        \item Automatische Erfassung durch Suchmaschinen
    \end{itemize}
    \item Nachteile
    \begin{itemize}[noitemsep,topsep=0pt,leftmargin=*]
        \item Aktualisierungs- und Konsistenzprobleme
        \item Keine Anwendungsfunktionalität
    \end{itemize}
\end{itemize}

\paragraph*{Dynamische Websites (Inhalt dynamisch generiert)} Verschiedene Ansätze:
\begin{itemize}[noitemsep,topsep=0pt,leftmargin=*]
    \item Datenbankorientierte Web-Applikation
    \item Verwendung von Webservices
    \item Clientseitiges Erzeugen der dynamischen Inhalte
\end{itemize}

\subsection{Verstehen von dynamischen Webseiten}
\paragraph*{Konzept}Das Problem mit statischen Websites ist, dass man mit der Seite selber nicht interagieren kann. Das ziel von dynamischen Websites ist es, eine \textbf{Benutzer-System-Interaktion} über das Web zu ermöglichen. Gewünschte Interaktionen beinhalten (nicht abschliessend):
\begin{itemize}[noitemsep,topsep=0pt,leftmargin=*]
    \item Suche - Ergebnis
    \item Eingabe - Speicherung
    \item Bestellung - Lieferung
    \item Profil Angabe - Personalisierung
\end{itemize}
\paragraph*{Datenbankorientiert}
\begin{itemize}[noitemsep,topsep=0pt,leftmargin=*]
    \item Server erstellt ein \textbf{Formular} zur Verfügung, das vom \textbf{Web-Browser angezeigt} wird
    \item Benutzer füllt Formular aus und drückt auf Submit Button
    \item Browser entnimmt die vom Benutzer eingegebenen Daten aus den entsprechenden Feldern und schick sie an den Server zusammen mit dem Namen
\end{itemize}

\subsection{Kennen der Arbeitsweise sowie der Vor- und Nachteile von nativen Applikationen, hybriden Applikationen und Web Applikationen}
\paragraph*{}

\subsection{Kennen der grundlegenden Adressierungsmechanismen}
\paragraph*{}

\end{document}