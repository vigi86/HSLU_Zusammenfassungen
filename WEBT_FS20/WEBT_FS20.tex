\documentclass[10pt,a4paper]{article}
\usepackage{ngerman}
\usepackage[utf8]{inputenc}
\usepackage{sectsty, xcolor}
\usepackage{lastpage}
\usepackage{amsmath, amsfonts, amssymb, enumitem, fancyhdr, graphicx, float, makeidx, textcomp, multicol}
\usepackage[hidelinks]{hyperref}
\usepackage[hang,flushmargin]{footmisc}
\usepackage{listings}
\lstloadlanguages{HTML}
\definecolor{pblue}{rgb}{0.13,0.13,1}
\definecolor{pgreen}{rgb}{0,0.5,0}
\definecolor{pred}{rgb}{0.9,0,0}
\definecolor{pgrey}{rgb}{0.46,0.45,0.48}
\definecolor{background}{rgb}{0.95,0.95,0.92}
\lstset{language=HTML,
  backgroundcolor=\color{background},
  frame=single,
  rulecolor=\color{pgrey},
  showspaces=false,
  showtabs=false,
  breaklines=true,
  showstringspaces=false,
  breakatwhitespace=true,
  commentstyle=\color{pgreen},
  keywordstyle=\color{pblue},
  stringstyle=\color{pred},
  breaklines=true,
  numbers=left,
  numberstyle=\tiny\color{pgrey},
  basicstyle=\ttfamily\footnotesize,
  inputencoding=utf8,
  extendedchars=true,
  literate={ä}{{\"a}}1 {ö}{{\"o}}1 {ü}{{\"u}}1,
  moredelim=[il][\textcolor{pgrey}]{\$\$},
  moredelim=[is][\textcolor{pgrey}]{\%\%}{\%\%}
}
\makeindex

\definecolor{dunkelblau}{rgb}{0,0.4,0.6}
\subsectionfont{\color{dunkelblau}}

\title{WEBT FS 2020}
\author{Victor Fernández}
\date{Februar 2020}

\addtolength{\oddsidemargin}{-.875in}
\addtolength{\evensidemargin}{-.875in}
\addtolength{\textwidth}{1.75in}
\addtolength{\topmargin}{-.875in}
\addtolength{\textheight}{1.75in}

% muss nach Änderung der margin kommen!
\pagestyle{fancy}
\fancyhf{} %reset
\fancyhead[L]{HSLU}
\fancyhead[C]{WEBT}
\fancyhead[R]{\thepage/\pageref{LastPage}}
\fancyfoot[L]{}
\fancyfoot[C]{}
\fancyfoot[R]{}
\renewcommand{\headrulewidth}{0.2pt} % Strich in Kopfzeile

\begin{document}

\maketitle
\tableofcontents
\thispagestyle{empty}
\pagebreak

\part{SW01 - Einführung}
\section{Inhalt}
\begin{itemize}[noitemsep,topsep=0pt,leftmargin=*]
    \item Klassifikation von Web-Applikationen
    \begin{itemize}[noitemsep,topsep=0pt,leftmargin=*]
        \item Statische Webseiten
        \item Dynamische Webseiten
        \begin{itemize}[noitemsep,topsep=0pt,leftmargin=*]
            \item Serverseitige Erzeugung: Datenbank-orientiert
            \item Serverseitige Erzeugung: mit Webservices
            \item Clientseitige Erzeugung
        \end{itemize}
    \end{itemize}
    \item Formen mobiler Applikationen
    \item Grundlegende Adressierungskonzepte
\end{itemize}

\subsection{Verstehen der verschiedenen Varianten von Web Applikationen}
\paragraph*{Statische Websites}Einfache statische Webauftritte, z.B. Homepage
\begin{itemize}[noitemsep,topsep=0pt,leftmargin=*]
    \item Vorteile
    \begin{itemize}[noitemsep,topsep=0pt,leftmargin=*]
        \item Einfache Basistechnologien: HTML und CSS
        \item Einfache Infrastruktur notwendig: Webserver
        \item Automatische Erfassung durch Suchmaschinen
    \end{itemize}
    \item Nachteile
    \begin{itemize}[noitemsep,topsep=0pt,leftmargin=*]
        \item Aktualisierungs- und Konsistenzprobleme
        \item Keine Anwendungsfunktionalität
    \end{itemize}
\end{itemize}

\paragraph*{Dynamische Websites (Inhalt dynamisch generiert)} Verschiedene Ansätze:
\begin{itemize}[noitemsep,topsep=0pt,leftmargin=*]
    \item Datenbankorientierte Web-Applikation
    \item Verwendung von Webservices
    \item Clientseitiges Erzeugen der dynamischen Inhalte
\end{itemize}

\subsection{Verstehen von dynamischen Webseiten}
\paragraph*{Konzept}Das Problem mit statischen Websites ist, dass man mit der Seite selber nicht interagieren kann. Das ziel von dynamischen Websites ist es, eine \textbf{Benutzer-System-Interaktion} über das Web zu ermöglichen. Gewünschte Interaktionen beinhalten (nicht abschliessend):
\begin{itemize}[noitemsep,topsep=0pt,leftmargin=*]
    \item Suche - Ergebnis
    \item Eingabe - Speicherung
    \item Bestellung - Lieferung
    \item Profil Angabe - Personalisierung
\end{itemize}
\paragraph*{Datenbankorientiert}
\begin{itemize}[noitemsep,topsep=0pt,leftmargin=*]
    \item Server erstellt ein \textbf{Formular} zur Verfügung, das vom \textbf{Web-Browser angezeigt} wird
    \item Benutzer füllt Formular aus und drückt auf Submit Button
    \item Browser entnimmt die vom Benutzer eingegebenen Daten aus den entsprechenden Feldern und schick sie an den Server zusammen mit dem Namen
\end{itemize}

\subsection{Kennen der Arbeitsweise sowie der Vor- und Nachteile von nativen Applikationen, hybriden Applikationen und Web Applikationen}
\paragraph*{}

\subsection{Kennen der grundlegenden Adressierungsmechanismen}
\paragraph*{}

\part{SW02 - HTML}
\section{HTML}
\begin{itemize}[noitemsep,topsep=0pt,leftmargin=*]
    \item Das W3C (World-Wide-Web Consortium) ist für Standardisierung zuständig
    \item Aktuellster fertiger Standard ist Version 5.2 (Dez. 2017)
    \item Zudem wurden verschiedene Themen rund um HTML5 in eigenen Spezifikationen verabschiedet, bzw. sind in der Entwicklung (AAM, HTML Extension Specifications, ARIA,\dots)
\end{itemize}
\subsection{Verstehen wie HTML Informationen strukturiert und wie der Aufbau von HTML Dokumenten ist}
\subsubsection{HTML tags}

\subsection{Wissen wie der \textbf{syntaktische Aufbau} von HTML ist}
\paragraph{$<$head$>$ }
\begin{itemize}[noitemsep,topsep=0pt,leftmargin=*]
    \item Enthält Kopfdaten wie Metainformation, Titel, Stil, Scriptdefinitionen, Adress- und Zielfensterbasis
    \item Ist in jedem HTML Dokument zu finden
    \begin{itemize}[noitemsep,topsep=0pt,leftmargin=*]
        \item Metainformationen werden durch Metatags repräsentiert
        \lstset{language=HTML}
        \begin{lstlisting}
<meta charset="utf-8">
<meta name="author" content="Hans Wurst">
        \end{lstlisting}
        \item Stildefinitionen (CSS - Cascaded Style Sheet) legen Darstellungen fest
        \begin{lstlisting}
<style>
    h1 { color: white; }
    p  { font-wight: bold; }
</style>
        \end{lstlisting}
    \end{itemize}
\end{itemize}

\paragraph{$<$body$>$}
\begin{itemize}[noitemsep,topsep=0pt,leftmargin=*]
    \item HTML \texttt{$<$body$>$} kennzeichnet den Anfang und das Ende des sichbaren Inhalts der WEbseite
    \item Browser zeigen nur den Inhalt zwischen dem öffnenden und schlissenden body-Tag im Browserfenster
    \item Enthält weitere HTML Tags welsche die Information strukturieren, aber auch \textbf{Scripts}, welche an der aufgeführten Stelle ausgeführt werden
    \item Ein HTML-Dokument darf nur einen body-Tag haben
\end{itemize}

\paragraph{Text- und Informations-Strukturierung}
\begin{itemize}[noitemsep,topsep=0pt,leftmargin=*]
    \begin{multicols}{2}
        \item Absatz
        \item Zeilenumbruch
        \item Vorformatierung
        \item Überschriften
        \item Waagrechte Linien
        \item Container
        \columnbreak
        \begin{lstlisting}
<p>..</p>
<br />
<pre>..</pre>
<h1>..</h1> bis <h6>..</h6>
<hr />
<div>..</div> oder <span>..</span>
        \end{lstlisting}
    \end{multicols}
    \item Sind alles \textbf{Blockelemente}, das heisst, ein neuer Absatz (Zeilenumbruch) wird eingeleitet
\end{itemize}

\paragraph{Verfügungen (Hypertext-Referenzen)}
\begin{itemize}[noitemsep,topsep=0pt,leftmargin=*]
    \item Link
    \begin{lstlisting}
<a href="pfad/datei">Linktext</a>
    \end{lstlisting}
    \item Sowohl lokal als auch ins Internet möglich
    \begin{lstlisting}
<a href="/index.html">Home</a>
<a href="http://www.hslu.ch">Gehe zu HSLU</a>
    \end{lstlisting}

    \item Mail-Links
    \begin{lstlisting}
<a href="mailto:hans@muster.ch">Mail schreiben</a>
    \end{lstlisting}
    \begin{itemize}[noitemsep,topsep=0pt,leftmargin=*]
        \item Sollte vermieden werden, da Spambots diese automatisiert auslesen
    \end{itemize}
        \item Interne Verknüpfung mittels Anker
    \begin{lstlisting}
<a href="#Kapitel1">Kapitel 1</a>
    \end{lstlisting}
    \item als Ziel dieses Links
    \begin{lstlisting}
<p id="Kapitel1">Kapitel 1</p>
    \end{lstlisting}
    \item Öffnen im neuen Fenster
    \begin{lstlisting}
<a href="adresse" target="_blank">Adresse</a>
    \end{lstlisting}
\end{itemize}

\paragraph{Grafiken}
\begin{itemize}[noitemsep,topsep=0pt,leftmargin=*]
    \item Grafikformate gif, jpg, png, \dots
    \item Einbinden mit
    \begin{lstlisting}
<img src="pfad/bildname" alt="Beschreibung" />
    \end{lstlisting}
    \item alt-Attribut verwenden spezielle Browser (Barrierefreiheit) oder Suchmaschinen (z.B. Google Bildersuche), unbedingt angeben
    \item Anzeigegrösse veränderbar mit Attributen \texttt{width} und \texttt{height}
    \item Rahmen: border="'1px"' %TODO
    \item Als Hintergrund der Seite
    \begin{lstlisting}
<body background="bildname">
    \end{lstlisting}
\end{itemize}

\paragraph{Klickbare Grafiken: Imagemaps}
\begin{enumerate}[noitemsep,topsep=0pt,leftmargin=*]
    \item Definition des Bildes
    \begin{lstlisting}
<map name="karte">
    <area shape="circle" coords="50,50,45" href="Ziel.html" alt="Reiseziel" />
</map>
    \end{lstlisting}
    \begin{itemize}[noitemsep,topsep=0pt,leftmargin=*]
        \item Wird häufig zu Navigationszwecken verwendet
    \end{itemize}
\end{enumerate}

\paragraph{Beispiel}Imagemap-Code
\begin{lstlisting}
<body>
    ...
    <img src="transmap.gif" alt="list of stations" usemap="#transmap">
    <map name="transmap">
        <area shape="rect" coords="59,390,172,441" href="#ACACIA" />
        <area shape="rect" coords="280,21,390,63" href="#ALMOND" />
        <area shape="rect" coords="135,52,243,124" href="#APPLE" />
        <area shape="rect" coords="141,235,189,284" href="#ASH" />
        <area shape="rect" coords="110,336,205,388" href="#BEECH" />
        <area shape="rect" coords="152,289,236,334" href="#BIRCH" />
        <area shape="rect" coords="330,231,402,288" href="#CHERRY" />
        ...
    </map>
    ...
    <a name="ACACIA">Acacia</a><img src="red.gif" alt="Red Line" />
    ...
    <a name="ALMOND">Almond</a><img src="yellow.gif" alt="Yellow Line" />
    ...
</body>
\end{lstlisting}

\paragraph{Metatags}
\begin{itemize}[noitemsep,topsep=0pt,leftmargin=*]
    \item Werden nicht zur Gestaltung sondern \textbf{zur Beschreibung des Inhalts} verwendet (daher der Name: Meta-Information)
    \item Werden im \texttt{head} Bereich eingefügt
    \item Aufbau:
    \begin{lstlisting}
<meta name="Schlüsselwort" content="Inhalt">
    \end{lstlisting}
    \item Die wichtigsten Metatags
    \begin{itemize}[noitemsep,topsep=0pt,leftmargin=*]
        \item keywords
        \item description
        \item language
        \item page-topic (Thema für Suchmaschinen und Kataloge)
        \item audience (Zielgruppe in Suchmaschinen und Katalogen)
        \item robots (zur Linkverfolgung)
        \item refresh (und expires $\rightarrow$Ablaufdatum)
        \item copyright
    \end{itemize}
\end{itemize}

\paragraph{Sonderzeichen}
\begin{itemize}[noitemsep,topsep=0pt,leftmargin=*]
    \item Damit Sonderzeichen korrekt dargestellt werden, muss das charset metatag korrekt gesetzt sein
    \begin{lstlisting}
<meta charset="utf8">
<meta charset="iso-8859-1">
    \end{lstlisting}
    \item Eine Alternative ist die Zeichen speziell zu kodieren:\\
    Beispiel: aus \textbf{ü} wird \textbf{\&uuml;}
    \item Dies ist auch notwendig bei Zeichen, welche mit dem HTML-Markup kollidieren:\\
    \& zu \&amp;, $<$ zu \&lt;, $>$ zu \&gt;, %TODO "` zu \&quot;, \quote zu \&#39;
\end{itemize}

\paragraph{Masseinheiten}
\begin{itemize}[noitemsep,topsep=0pt,leftmargin=*]
    \item Verwendet in Attributen zur Bestimmung der Dimensionen verschiederer Elemente wie Bilder, Schriften, Ränder, Abstände
    \item Die wichtigsten Einheiten sind:
    \begin{itemize}[noitemsep,topsep=0pt,leftmargin=*]
        \item pt: Punkt, \textbf{absolute} Angabe, 1 Punkt entspricht 1/72 Inches
        \item in: Inch, \textbf{absolute} Angabe, 1 Inch enspricht 2.54cm
        \item mm: Millimeter, \textbf{absolute} Angabe
        \item px: Pixel, \textbf{absolute/relative} Angabe Abhängig von der \textbf{Pixeldichte} des Ausgabegeräts
        \item em: M, \textbf{relative} Angabe, auf die Schriftgrösse des Elements bezogen (mit Ausnahmen)
        \item \%: Prozent, \textbf{relative} Angabe, je nach CSS-Eigenschaft relativ zur elementeigenen Grösse, oder zu der des Elternelements, oder zu einem allgemeineren Kontext
    \end{itemize}
\end{itemize}

\paragraph{Farben}
\begin{itemize}[noitemsep,topsep=0pt,leftmargin=*]
    \item Farben werden aus den RGB-Wertangaben gebildet
    \item Beispiel: \#FF0000 ist rot, \#00FF00 ist grün, \#0000FF ist blau
    \item Werte werden in hexadezimaler Form angegeben
    \item Werte von \#00 bis \#FF (255) sind möglich
    \item Einige Farben sind per Namen in der DTD (Document-Type-Definition) definiert:
    \begin{align*}
&\text{Black}&      &=\text{\#00000}&   &\text{Green}&   &=\text{\#008000}\\
&\text{Silver}&     &=\text{\#C0C0C0}&  &\text{Lime}&    &=\text{\#00FF00}\\
&\text{Gray}&       &=\text{\#808080}&  &\text{Olive}&   &=\text{\#808000}\\
&\text{White}&      &=\text{\#FFFFFF}&   &\text{Yellow}&  &=\text{\#FFFF00}\\
&\text{Maroon}&     &=\text{\#800000}&   &\text{Navy}&    &=\text{\#000080}\\
&\text{Red}&        &=\text{\#FF0000}&   &\text{Blue}&    &=\text{\#0000FF}\\
&\text{Purple}&     &=\text{\#800080}&   &\text{Teal}&    &=\text{\#008080}\\
&\text{Fuchsia}&    &=\text{\#FF00FF}&   &\text{Aqua}&    &=\text{\#00FFFF}\\
    \end{align*}
\end{itemize}

\subsubsection{Struktur von Webseiten - Elemente}
\begin{figure}[H]
    \begin{center}
    \includegraphics[width=14cm]{images/struktur.png}
    \caption{Strukturbeispiel von Webseiten}
    \label{strukturHTML}
    \end{center}
\end{figure}

\paragraph{$<$header$>$}
\begin{itemize}[noitemsep,topsep=0pt,leftmargin=*]
    \item enthält sichtbaren Kopfbereich einer Webseite
    \item Gruppierung einleitender Inhalte (Firmenlogos, Motto, Navigationslinks)
\end{itemize}

\paragraph{$<$footer$>$}
\begin{itemize}[noitemsep,topsep=0pt,leftmargin=*]
    \item enthält Informationen, die in Webseiten am Ende stehen: Autor, Hinweise zum Urheberrecht, ein Link zum Impressum
    \item Position ist nicht notwendigerweise am unteren Rand\\
    $\rightarrow$bei Blogeinträgen steht der footer oft neben dem Text
\end{itemize}

\paragraph{$<$article$>$}
\begin{itemize}[noitemsep,topsep=0pt,leftmargin=*]
    \item stellt in sich geschlossene Abschnitte eines Dokuments dar\\
    $\rightarrow$vergleichbar mit einem Zeitungsartikel\\
    $\rightarrow$innterhalb von article-Elementen weitere strukturierende Elemente wie header, section oder footer
\end{itemize}

\paragraph{$<$section$>$}
\begin{itemize}[noitemsep,topsep=0pt,leftmargin=*]
    \item enthält eine thematische Gruppierung von Inhalten typischerweise mit einer Überschrift
    \item dient dazu, den Inhalt oder auch einen article in semantische Abschnitte zu gliedern
\end{itemize}

\paragraph{$<$nav$>$}
\begin{itemize}[noitemsep,topsep=0pt,leftmargin=*]
    \item umschliesst insbesondere Navigationsleisten
    \item kann neben einer ungeordneten Liste mit den Verweisen auch eine Überschrift oder ähnliches enthalten
\end{itemize}

\paragraph{$<$aside$>$}
\begin{itemize}[noitemsep,topsep=0pt,leftmargin=*]
    \item umschliesst Abschnitte einer Seite, deren Inhalt nur in einem indirekten Zusammenhang mit dem umgebenden Inhalt stehen
    \item Beispiele: Randbemerkungen, Fussnoten oder Links zu weitergehenden Webseiten
\end{itemize}


\subsection{Kennen von \textbf{geeigneten Werkzeugen} für das Erstellen, Bearbeiten, Darstellen, Validieren etc. von HTML Dokumenten}

\subsection{Wissen um geeignete \textbf{Quellen und Referenzen} im Internet}

\part{SW04 - CSS}
\section{Inhalt}
\subsection{Wissen, wie die \textbf{Kaskadenkette} zur schlussendlichen Darstellungsform führt}
\paragraph{Kaskadierung}Falls mehrere, sich eventuell wiedersprechnde Styles definiert sind, muss ein Regelwerk zur Anwendung kommen um zu einem Ergebnis zu gelangen.
\noindent
Grundprinzip dieses Regelwerkes: $\rightarrow$Prioritätsreihenfolge:
\begin{itemize}[noitemsep,topsep=0pt,leftmargin=*]
    \item Gewichtung (Schlüsselwort \texttt{!important})
    \item Herkunft (Web-Author, Benutzer, Browser)
    \item Besonderheit (je spezifischer desto mehr Gewicht)
    \item Ergebnisabfolge (Reihenfolge in der die Definitionen festgelegt wurden)
\end{itemize}

\paragraph{Prioritätenreihenfolge nach Herkunft}
\begin{enumerate}[noitemsep,topsep=0pt,leftmargin=*]
    \item Falls Deklarationnen des Browser
    \item Deklarationen des Benutzers\footnote{Benutzer$\Rightarrow$Webseite Besucher$\rightarrow$also seine Browsereinstellungen}
    \item Interne/Externe CSS-Anweisungen des Web-Authors
    \item Inline CSS-Anweisungen des Web-Authors
    \item Deklarationen des Web-Authors die \texttt{!important} enthalten
    \item Deklarationen des Benutzers\textsuperscript{1} die \texttt{!important} enthalten
\end{enumerate}
\begin{itemize}[noitemsep,topsep=0pt,leftmargin=*]
    \item Der Vorrang der \textbf{!important-Benutzer}- gegenüber den \textbf{!important-Autoren}- Styles wurde mit dem CSS2-Standard neu festgelegt.
\end{itemize}

\paragraph{Unterschiedliche Ausgabemedien}
\begin{itemize}[noitemsep,topsep=0pt,leftmargin=*]
    \item Es ist möglich für verschiedene Ausgabemedien Stylesheets zu definieren
    \item Dadurch können die verschiedenen Charakteristiken der Ausgabemedien (Drucker, Screen, Sprachausgabe, etc.) besser berücksichtigt werden
    \item Das Attribut \texttt{media} definiert, für welches Ausgabemedium der entsprechende Style definiert ist
\end{itemize}

\paragraph{Unterschiedliche Ausgabemedien}
\begin{itemize}[noitemsep,topsep=0pt,leftmargin=*]
    \item Beispiel, Ausgabe für Bildschrim oder Drucker (hier im HTML-Code)
\end{itemize}
\begin{lstlisting}
<head>
    <title>Seitentitel</title>
    <link href="standard.css" rel="stylesheet" media="screen" title="Standard-Layout">
    <link href="druck.css" rel="stylesheet" media="print" title="Druckoptimiertes Layout">
    <link href="aural.css" rel="stylesheet" media="speech" title="Sprachausgabe">
</head>
\end{lstlisting}

\paragraph{Unterschiedliche Ausgabemedien}
\begin{itemize}[noitemsep,topsep=0pt,leftmargin=*]
    \item standard.css\\ist das Standard-Stylesheet für die Bildschrimanzeige
    \item print.css\\ist das Standard-Stylesheet für den Ausdruck
    \item aural.css\\ist für zukünftige Definitionen bezüglich der Sprachausgabe vorgesehen. Z.B. Sprechgeschwindigkeit, männliche oder weibliche Stimme, etc.
\end{itemize}

\paragraph{Schriftformatierung: Grundsätzliches}
\begin{itemize}[noitemsep,topsep=0pt,leftmargin=*]
    \item definiert Schriftart bzw. -typ\\
    \texttt{font-family}
    \item Beispiel:\\
    \verb|font-family:"Times New Roman";|
    \item Es kann sein, dass diese Schriftart auf dem Rechner des Benutzers nicht vorhanden ist, deshalb lassen sich Alternativen angeben:\\
    \verb|font-family:"Times New Roman", Garamond, serif;|
    \item Möchte man nur den Schrifttyp angeben, sucht das System des Benutzers die passende Schriftart dazu aus:\\
    \texttt{font-family:sans-serif;}
\end{itemize}

\paragraph{Schriftformatierung: Schriftgrösse und -neigung}
\begin{itemize}[noitemsep,topsep=0pt,leftmargin=*]
    \item Fast jede Eigenschaft der Schriftart lassen sich steuern, sei es Grösse, Neigung, Dicke, etc.
    \item \texttt{font-size} legt die Grösse fest, z.B. absolut: \texttt{font-size:20px;} oder relativ: \texttt{font-size:small;}
    \item Die relativen Angaben sind abhängig vom OS, Browser oder anderen Einstellungen
    \item \texttt{font-style} steuert die Neigung, \texttt{font-variant} die Varianten der Schrift: \texttt{font-style:italic;} ergibt eine kursive Darstellung, \texttt{font-variant:small-caps;} zeigt Kapitälchen\footnote{Kapitälchen: die Kleinbuchstaben werden als grosse Buchstaben dargestellt, doch in der höhe von Kleinbuchstaben}
\end{itemize}

\paragraph{Schriftformatierung: Schriftdicke und -farbe}
\begin{itemize}[noitemsep,topsep=0pt,leftmargin=*]
    \item \texttt{font-weight} legt die Schriftdicke fest. Es sind sowohl absolute als auch relative Angaben möglich: \texttt{font-weight:bold;} oder \texttt{font-weight:600}
    \item \textbf{Achtung:} nicht jede Schriftart unterstützt diese Angaben!
    \item Die Eigenschaft \texttt{color} verändert die Farbe. es können Farbworte oder Tripelwerte verwendet werden:
    \begin{itemize}[noitemsep,topsep=0pt,leftmargin=*]
        \item \texttt{color:black;}
        \item \texttt{color:\#88AAFF;}
        \item \texttt{color:rgb(23\%,50\%,95\%);}
    \end{itemize}
\end{itemize}

\paragraph{Schriftformatierung: Textdekoration}
\begin{itemize}[noitemsep,topsep=0pt,leftmargin=*]
    \item CSS versteht u.A. folgende Arten der Textdekoration: Über-, Durch- oder Unterstrichen
    \item Die Eigenschaft \texttt{text-decoration} steuert die Darstellung:
    \begin{itemize}[noitemsep,topsep=0pt,leftmargin=*]
        \item \texttt{text-decoration:underline;}
        \item \texttt{text-decoration:overline;}
        \item \texttt{text-decoration:line-through;}
    \end{itemize}
    \item Weitere Möglichkeiten:
    \begin{itemize}[noitemsep,topsep=0pt,leftmargin=*]
        \item \texttt{text-decoration:blink;} (zu vermeiden!)
        \item \texttt{text-decoration:none;}
    \end{itemize}
\end{itemize}

\paragraph{Schriftformatierung: Texttransformation}
\begin{itemize}[noitemsep,topsep=0pt,leftmargin=*]
    \item Die Eigenschaft \texttt{text-transform} verändert die Gross- und Kleinschreibung eines Textes:
    \begin{itemize}[noitemsep,topsep=0pt,leftmargin=*]
        \item \texttt{text-transform:uppercase;}
        \item \texttt{text-transform:lowercase;}
        \item \texttt{text-transform:normal;}
        \item \texttt{text-transform:capitalize}
    \end{itemize}
    \item \texttt{capitalize} stellt alle Wortanfänge mit Grossbuchstaben dar
\end{itemize}

\paragraph{Schriftformatierung: Kurzform der Schriftformatierungen}
\begin{itemize}[noitemsep,topsep=0pt,leftmargin=*]
    \item Um effizienter die Schriftformatierung festzulegen, können die Eigenschaften \texttt{font-family, font-size, font-variant} und \texttt{font-weight} zusammengefasst in der \texttt{font} Eigenschaft angegeben werden
    \item Beispiele:
    \begin{itemize}[noitemsep,topsep=0pt,leftmargin=*]
        \item \texttt{font:italic 14px Arial;}
        \item \texttt{font:lighter 12pt monospace;}
    \end{itemize}
\end{itemize}

\paragraph{Attribut-bedingte Formatierung}
\begin{itemize}[noitemsep,topsep=0pt,leftmargin=*]
    \item CSS-Formatierungen lassen sich auf Elemente mit bestimmten Attributen begrenzen
    \item Beispiele:
    \begin{itemize}[noitemsep,topsep=0pt,leftmargin=*]
        \item legt Farbe für alle h1 Überschriften fest, welche ein Alignment haben\\
        \verb|{h1[align] {color:blue}|
        \item stellt alle Absätze, die ein Namenattribut mit Inhalt "`Text"' haben, mit Kapitälchen dar\\
        \verb|p[name*="Text"] {font-variant:small-caps;}|
        \item weist allen zentriert ausgerichteten HTML-Elementen eine Farbe zu\\
        \verb|*[align=center] {color:red;}|
    \end{itemize}
\end{itemize}

\paragraph{Individuelle CSS-Formate}
\begin{itemize}[noitemsep,topsep=0pt,leftmargin=*]
    \item Um Elementen individuelle Formate zu geben, verwendet man die individuelle Formatierung
    \item Jedes Element kann ein \textbf{eindutiges} Attribut \textbf{id} besitzen (Element-ID)
    \begin{itemize}[noitemsep,topsep=0pt,leftmargin=*]
        \item Element-IDs kann man auch für JavaScript brauchen!
    \end{itemize}
    \item Beispiel:\\
    \verb|#eins {color:blue;}|\\
    \verb|<p id="eins">Absatz</p>|
\end{itemize}

\paragraph{Hintergrundbilder}
\begin{itemize}[noitemsep,topsep=0pt,leftmargin=*]
    \item Hintergrundbilder einer Seite lassen sich folgendermassen festlegen:\\
    \verb|body {background-image:url(bild.jpg);}|
    \item Man kann aber auch nur für einen Abschnitt ein Hintergrundbild festlegen:\\
    \verb|p {background-image:url(bild.jpg);}|
    \item Je nach Grösse kann das Bild mehrfach wiederholt werden, sowohl horizontal als auch vertikal:
    \begin{lstlisting}
table {
    background-image:url(bild.jpg);
    background-repeat:repeat-x;
    background-repeat:repeat-y;
}
    \end{lstlisting}
\end{itemize}

\paragraph{Zentrieren von Containern}
\begin{itemize}[noitemsep,topsep=0pt,leftmargin=*]
    \item Weil das \texttt{align} Attribut nur für Textabschnitte gilt, kann es \textbf{nicht} zum zentrieren von \texttt{$<$div$>$} Containern bzw. neuen HTML5 Block-Elementen verwendet werden
    \item Man verwendet dazu die CSS Eigenschaft \texttt{margin}
    \item Beispiel:
    \verb|<div style="margin:auto">Containerinhalt</div>|
\end{itemize}

\paragraph{Fliesstext - float}
\begin{itemize}[noitemsep,topsep=0pt,leftmargin=*]
    \item Mittels der \texttt{float} Eigenschaft kann man Text um Bereiche herum fliessen lassen (oder verbieten)
    \item Beispiel:
    \begin{lstlisting}
<p>
    <span style="color:red; float:left; font-size:2.5em; padding-right:2px;">D</span>as ist eine umfliessende Initiale, in der nur der erste Buchstabe eines Textes gross geschrieben wird
</p>
    \end{lstlisting}
\end{itemize}

\paragraph{Mehrspaltiges Layout: Definition}\lstinline|<p class="blub">hallo</p>|

\end{document}