\documentclass[10pt,a4paper]{article}
\usepackage{ngerman}
\usepackage[utf8]{inputenc}
\usepackage{sectsty, xcolor}
\usepackage{lastpage}
\usepackage{amssymb, enumitem, fancyhdr, graphicx, float, makeidx, textcomp, multicol}
\usepackage[hidelinks]{hyperref}
\usepackage[hang,flushmargin]{footmisc}
\makeindex

\definecolor{dunkelblau}{rgb}{0,0.4,0.6}
\subsectionfont{\color{dunkelblau}}

\title{DMATH FS 2020}
\author{Victor Fernández}
\date{Januar 2020}

\addtolength{\oddsidemargin}{-.875in}
\addtolength{\evensidemargin}{-.875in}
\addtolength{\textwidth}{1.75in}
\addtolength{\topmargin}{-.875in}
\addtolength{\textheight}{1.75in}

% muss nach Änderung der margin kommen!
\pagestyle{fancy}
\fancyhf{} %reset
\fancyhead[L]{HSLU}
\fancyhead[C]{DMATH}
\fancyhead[R]{\thepage/\pageref{LastPage}}
\fancyfoot[L]{}
\fancyfoot[C]{}
\fancyfoot[R]{}
\renewcommand{\headrulewidth}{0.2pt} % Strich in Kopfzeile

\begin{document}

\maketitle
\tableofcontents
\thispagestyle{empty}
\pagebreak

\section{Logik}
\paragraph{Propositionen (Aussagen)}Eine Proposition ist ein Satz, der entweder wahr (Wahrheitswert w) oder falsch (Wahrheitswert f) ist.
\paragraph{Negation}Ist $p$ eine Propostion, dann ist die Proposition "`Es ist nicht der Fall, dass p gilt"' die Negation von $p$; man schreibt $\neg p$ und liest "`nicht p"'.
\paragraph{Wahrheitstabelle}Die Wahrheitstabelle stellt die Beziehungen zwischen den Wahrheitswerten von Propositionen dar. Sie ist vor allem dann nützlich, wenn Propositionen aus einfachen Propositionen konstruiert werden.\\
\begin{tabular}{|c|c|}
    \hline
        $p$&$\neg p$\\
        \hline
        w&f\\
        f&w\\
    \hline
\end{tabular}
\paragraph{Konjunktion - UND-Verknüpfung}Die Propositionen $p\wedge q$ (gelesen "`p und q"') heisst Konjunktion der Propositionen p und q, falls diese genau dann wahr ist, wenn p und q wahr sind; andernfalls ist sie falsch.
\paragraph{Disjunktion - ODER-Verknüpfung}Die Propositionen $p\vee q$ (gelesen "`p oder q"') heisst Disjunktion der Propositionen p und q falls diese wahr ist, wenn mindestens eine der Propositionen p oder q wahr ist; andernfalls ist sie falsch.
\paragraph{Konjunktion und Disjunktion}UND- und ODER-Verknüpfung\\
\begin{tabular}{|c|c|c|c|}
    \hline
        $p$&$q$&$p\wedge q$&$p\vee q$\\
        \hline
        w&w&w&w\\
        w&f&f&w\\
        f&w&f&w\\
        f&f&f&f\\
    \hline
\end{tabular}
\paragraph{XOR-Verknüpfung (eXklusives OR, EXOR)}

\end{document}