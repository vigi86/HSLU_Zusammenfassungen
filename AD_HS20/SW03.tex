\part{SW 03 - Bäume}
\section{Lernziele}
\begin{itemize}
    \item Sie wissen wie eine baumartige Datenstruktur aufgebaut ist
    \item Sie kennen verschiedene Beispiele von Baumstrukturen
    \item Sie kennen die Grundelemente eines Baumes:\\
    Wurzel, Knoten, Blatt und Kanten
    \item Sie können die Kenngrössen eines Baumes beschreiben
\end{itemize}

\subsection{Verwendung und Arten von Baumstrukturen}
Zwei grundlegende Szenarien:
\begin{enumerate}
    \item Die Daten haben bereits inhärent eine hierarchische Struktur, welche man entsprechend abbilden will. Beispiele:
    \begin{itemize}
        \item Dateisystem mit Verzeichnissen und Dateien
        \item Stammbaum (Genealogie)
        \item Vererbungshierarchie in Java (nur mit Einfachvererbung)
    \end{itemize}
    \item Wenn man in einer geordneten Datenmenge einzelne Elemente sehr schnell finden will $\rightarrow$binärer Suchbaum
    \begin{itemize}
        \item Die Suche über eine Baumstruktur hat typisch nur einen Aufwand von $\mathcal{O}(\log n)$, und ist somit der rein sequenziellen Suche mit \textbf{$\mathcal{O}(n)$} deutlich überlegen
    \end{itemize}
\end{enumerate}
\begin{itemize}
    \item Mit Ausnahme der Wurzel (Ursprung des Baumes, die \textbf{alle} baumartigen Strukturen haben) können Bäume sehr stark variieren:
    \begin{itemize}
        \item Unterschiedliche Anzahl Äste
        \item Unterschiedliche Länge (Tiefe) der Äste
        \item Die Breite (Grad) und die Höhe der Bäume ist sehr variabel
    \end{itemize}
    \item Je nach Anwendungszweck definiert man mehr oder weniger \textbf{Restriktionen}, welche dann zu spezifischeren Baumstrukturen führen, welche auch spezifischere Eigenschaften aufweisen
    \begin{itemize}
        \item Zweks Beschleunigung und/oder einfacherer Algorithmen
    \end{itemize}
\end{itemize}

\subsection{Gerichtete und ungerichtete Bäume}
\begin{itemize}
    \item Ein ungerichteter Baum ist eine reine Hierarchie
    \item Out-Tree, Navigation von der Wurzel \textbf{nach unten} zu den Blättern
    \begin{itemize}
        \item $\rightarrow$Kanten (Pfeile) gehen von der Wurzel aus. Am Häufigsten!
    \end{itemize}
    \item In-Tree, Navigation von den Blättern \textbf{nach oben} zur Wurzel
    \begin{itemize}
        \item $\rightarrow$Kanten (Pfeile) zeigen zur Wurzel hin. Seltener.
    \end{itemize}
    \item Diverse Spezialformen von Bäumen (Beispiele)
    \begin{itemize}
        \item \textbf{Binär}-Baum - am einfachsten und häufigsten
        \item \textbf{AVL}-Baum - höhenbalancierter Binärbaum
        \item \textbf{B}-Baum - balancierter Baum, \textbf{nicht} zwingend binär!
        \item \textbf{B*}-Baum - restriktivere Form B-Baumes (ebenfalls balanciert)
        \item \textbf{Binomial}-Baum - speziell strukturierter Baum
        \item etc.
    \end{itemize}
\end{itemize}

\subsection{Kenngrössen von Bäumen}
\subsubsection{Ordnung}
\begin{itemize}
    \item Die \textbf{Ordnung} (order) eines Baumes definiert, wie viele Kinder ein Knoten \textbf{maximal} haben darf
    \begin{itemize}
        \item Die Anzahl muss in eine konkreten (Teil-)BAum aber \textbf{nicht} zwingend erreicht werden
    \end{itemize}
    \item Die Ordnung ist eine Definition!
\end{itemize}

\subsubsection{Grad}
\begin{itemize}
    \item Der \textbf{Grad} (degree) eines Knotens sagt, wie viele Kinder ein bestimmter Knoten \textbf{aktuell} tatsächlich hat
    \item Bei einem Baum, z.B. der \textbf{fünften} Ordnung, darf der Grad jedes einzelnen Knotens \textbf{maximal 5} betragen, also maximal fünf Kinder
\end{itemize}

\subsubsection{Pfad}
\begin{itemize}
    \item Als \textbf{Pfad} (path) eines Knotens bezeichnet man den Weg von der Wurzel bis zum entsprechenden Knoten, bzw. Blatt
\end{itemize}

\subsubsection{Tiefe}
\begin{itemize}
    \item Die \textbf{Tiefe} (depth) eines Knotens beschreibt die Länge seines Pfades. Dazu werden die Kanten auf seinem Pfad gezählt
    \item Achtung: Es gibt auch eine Zählweise die bei 1 beginnt; es ist nicht einheitlich geregelt
\end{itemize}

\subsubsection{Niveaus / Ebenen (levels)}
\begin{itemize}
    \item Als \textbf{Niveau} oder \textbf{Ebene} bezeichnet man die Menge aller Knoten, welche die gleiche \textbf{Tiefe} haben
\end{itemize}

\subsubsection{Höhe}
\begin{itemize}
    \item Die \textbf{Höhe} (height) eines Baumes definiert sich aus der \textbf{Tiefe} des Knotens, welcher am \textbf{weitesten} von der Wurzel entfernt ist, bzw. aus der Anzahl der \textbf{Niveaus}
\end{itemize}

\subsubsection{Gewicht}
\begin{itemize}
    \item Das \textbf{Gewicht} (weight) eines Baumes definiert sich über die Anzahl der enthaltenen Knoten
\end{itemize}

\subsection{Füllgrade}
\subsubsection{Ausgefüllt}
\begin{itemize}
    \item Ein Baum wird als \textbf{ausgefüllt} bezeichnet, wenn \textbf{jeder innere Knoten} die \textbf{maximale} Anzahl an Kindern hat
    \item Der \textbf{Grad aller} inneren Knoten ist somit \textbf{gleich} der \textbf{Ordnung} des Baumes
\end{itemize}

\subsubsection{Voll}
\begin{itemize}
    \item Ein Baum wird als \textbf{voll} bezeichnet, wenn das \textbf{letzte} Niveau linksbündig (oder auch rechts) angeordnet ist, und alle \textbf{restlichen} Niveaus die \textbf{maximale} Anzahl an Kindern enthalten
\end{itemize}

\subsubsection{Vollständig oder komplett}
\begin{itemize}
    \item Ein Baum wird als \textbf{vollständig} oder \textbf{komplett} bezeichnet, wenn \textbf{jedes} Niveau die \textbf{maximale} Anzahl Knoten enthält
    \begin{itemize}
        \item Er hat dann für sein \textbf{Gewicht} die \textbf{minimale} Anzahl \textbf{Niveaus}
        \item Die Struktur ist immer \textbf{symmetrisch} und ausgeglichen
    \end{itemize}
\end{itemize}

\section{Binäre Bäume}
\subsection{Lernziele}
\begin{itemize}
    \item Sie sind mit binären Bäumen vertraut
    \item Sie kennen die Algorithmen, um binäre Bäume auf unterschiedliche Arten zu traversieren
    \item Sie sind mit den spezielen Eigenschaften von binären Suchbäumen vertraut
    \item Sie wissen wie das Suche, Einfügen und Entfernen von Knoten in binären Suchbäumen konzeptionell abläuft
    \item Sie verstehen, was ein ausgeglichener Baum ist und wie man diesen Zustand herstellen kann
\end{itemize}

\subsection{Binärer Baum}
\begin{itemize}
    \item Ein \textbf{binärer Baum} (binary tree) ist als Baum mit \textbf{Ordnung 2} definiert. Jeder Knoten hat somit maximal \textbf{zwei} Kinder
    \begin{itemize}
        \item Diese werden als \textbf{linker und rechter} Kindknoten bezeichnet
    \end{itemize}
    \item Binäre Bäume sind in der Informatik \textbf{sehr} beliebt, weil:
    \begin{itemize}
        \item durch die Beschränkung auf die \textbf{Ordnung 2} einige Algorithmen stark vereinfacht werden
        \item auf binären Bäumen unterschiedliche Durchlaufordnungen ($\rightarrow$Traversierungen) möglich sind
        \item die Suche in einem binären (Such-)Baum einer binären Suche entspricht
    \end{itemize}
\end{itemize}

\subsection{Traversieren eines binären Baumes}
\begin{itemize}
    \item Aufgrund der spezifischen Eigenschaft von binären Bäumen (Ordnung 2) kann man diese auf \textbf{drei} unterschiedliche Arten traversieren (vergleiche dazu Iteration bei \textbf{Listen})
    \begin{itemize}
        \item \textbf{Preorder} - Hauptreihenfolge
        \item \textbf{Postorder} - Nebenreihenfolge
        \item \textbf{Inorder} - Symmetrische Reihenfolge
    \end{itemize}
    \item Die Algorithmen, welche diese drei verschiedenen Traversierungsarten beschreiben, sind alle \textbf{rekursive} Algorithen
    \item Alle Traversierungen sind direkt abhängig von der Anzahl Knoten und haben somit einen Aufwand von $\mathcal{O}(n)$
\end{itemize}

\subsubsection{Preorder}
\subsubsection{Postorder}
\subsubsection{Inorder}

\subsection{Binäre Suchbäume}
\subsection{Geordneter binärer Suchbaum}
