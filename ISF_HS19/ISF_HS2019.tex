\documentclass[10pt,a4paper]{article}
\usepackage{ngerman}
\usepackage{graphicx}
\usepackage{fancyhdr}
\usepackage{lastpage}
\usepackage[hidelinks]{hyperref}
%\usepackage{authblk}

\title{ISF HS 2019}
\author{Victor Fernández}
\date{Dezember 2019}


\addtolength{\oddsidemargin}{-.875in}
\addtolength{\evensidemargin}{-.875in}
\addtolength{\textwidth}{1.75in}

\addtolength{\topmargin}{-.875in}
\addtolength{\textheight}{1.75in}


\pagestyle{fancy}
\fancyhf{} %reset
\fancyhead[L]{HSLU}
\fancyhead[C]{ISF}
\fancyhead[R]{\thepage/\pageref{LastPage}}
\fancyfoot[L]{}
\fancyfoot[C]{}
\fancyfoot[R]{}
\renewcommand{\headrulewidth}{0.2pt}

%******
\begin{document}
\maketitle
\thispagestyle{empty}
\tableofcontents
\thispagestyle{empty}
\pagebreak

\part{Einführung (SW~01)}
\section{Einführung}
\subsection*{Einführung in das Thema "`Management von Informationssicherheit"'}
\paragraph*{Daten, Information und Wissen}
Information ist die Verknüpfung von Daten in Form von Zahlen, Worten und Fakten zu interpretierbaren Zusammenhängen.
Durch die Vernetzung von Informationen entsteht Wissen, das zunächst personenbezogen ist.
\paragraph*{Missbrauch}Informationen müssen vor Missbrauch geschützt werden
\begin{figure}[h]
    \begin{center}
    \includegraphics[width=8cm]{images/Wissenspyramide.png}
    \caption{Wissenspyramide}
    \label{Wissenspyramide}
    \end{center}
\end{figure}
\subsection*{Motivation / Bedrohungen}
\paragraph*{TODO}
\subsection*{Grundbegriffe}
\textbf{Zutritts-, Zugangs-, Zugriffskontrolle}\\
\textbf{\textsl{Zutrittskontrolle: }}Schutz des physischen Systems (Bsp. Serverraum)\\
\textbf{\textsl{Zugangskontrolle: }}Schutz des logischen Systems (Bsp. Betriebssystem)\\
\textbf{\textsl{Zugriffskontrolle: }}Daten-bezogen; Schutz der Operationen (Bsp. Dateisystem)

\part{Kryptographie (SW~02-04)}
\section{Symmetrische Kryptographie}
\subsection*{Sie verstehen was Steganographie ist}
\paragraph*{TODO}
\subsection*{Sie verstehen was Private-Key-Kryptographie ist, welche Arten von Sicherheit es gibt und welche Angriffsarten auf Verschlüsselung existieren}
\paragraph*{TODO}
\subsection*{Sie können "`klassische"' symmetrische Verschlüsselungverfahren wie Ceasar cipher, Vigenère cipher, one-time pad anwenden und verstehen die Vor- und Nachteile bzw. Schwachstellen dieser Verfahren}
\paragraph*{TODO}
\subsection*{Sie wissen welche modernen Verschlüsselungsalgorithmen in der Praxis verwendet werden und was deren Eigenschaften sind}
\paragraph*{TODO}
\subsection*{Sie verstehen was eine Hashfunktion ist und welche Eigenschaften eine kryptographische Hashfunktion ausmachen, bzw. was es heisst, wenn eine Hashfunktion gebrochen ist}
\paragraph*{TODO}
\subsection*{Sie kennen moderne Hashfunktionen und wissen welche Eigenschaften diese haben}
\paragraph*{TODO}
\subsection*{Sie kennen Anwendungen von Hashfunktionen}
\paragraph*{TODO}
\subsection*{Sie wissen was ein keyed Hash (HMAC) ist und wofür dieser verwendet werden kann}
\paragraph*{TODO}
\subsection*{Sie kennen die "`Best-practices"' zu Passwortsicherheit und wissen, gegen welche Angriffe diese schützen}
\paragraph*{TODO}

\section{Asymmetrische Kryptographie}
\subsection*{Sie verstehen was Public-Key-Kryptographie ist, worauf deren Sicherheit basiert und wie sie zur Verschlüsselung, für Signaturen und zur Authentisierung verwendet werden kann}
\paragraph*{TODO}
\subsection*{Sie kennen die gängigen asymmetrischen Verschlüsselungs- und Signaturalgorithmen und wissen, worauf deren Sicherheit basiert}
\paragraph*{TODO}
\subsection*{Sie wissen wie Diffie-Hellmann-Schlüsselaustausch bzw. ElGamal-Verschlüsselung funktioniert}
\paragraph*{TODO}
\subsection*{Sie wissen was kryptographisch sichere Zufallszahlen sind und wo diese verwendet werden}
\paragraph*{TODO}
\subsection*{Sie wissen was eine elektronische Signatur ausmacht}
\paragraph*{TODO}
\subsection*{Sie wissen wie hybride Verschlüsselung bzw hybride Signaturen funktionieren}
\paragraph*{TODO}

\section{Zertifikate und SSL-TLS}
\subsection*{Sie kennen die verschiedenen Arten von "`Trust"'}
\paragraph*{TODO}
\subsection*{Sie wissen was eine Public-Key-Infrastruktur, eine Certificate Authority und ein Zertifikat ist, wofür und wie diese verwendet werden und wie Zertifikate ausgestellt und revoziert werden}
\paragraph*{TODO}
\subsection*{Sie wissen was SSL/TLS ist, welche Funktionalität es erreicht und wie das Protokoll konzeptionelle abläuft}
\paragraph*{TODO}

\part{Angriffe (SW~05-06)}
\section{Angriffe auf Webanwendungen}
\paragraph*{TODO}
\subsection*{Sie wissen was eine Webanwendung ausmacht, wie HTTP funktioniert}
\paragraph*{TODO}
\subsection*{Sie wissen was eine Session ist und welche Eigenschaften einer Session bei welchen Angriffen wichtig sind bzw wie sie gegen gewisse Angriffe Schutz bieten}
\paragraph*{TODO}
\subsection*{Sie kennen sicherheitsrelevante Header}
\paragraph*{TODO}
\subsection*{Sie verstehen wie ein Cross-Site-Request-Forgery-Angriff abläuft und wie man sich dagegen schützen kann}
\paragraph*{TODO}

\section{Angriffe auf Protokollebene}
\subsection*{Sie kennen die Grundbegriffe der Anwendungssicherheit}
\paragraph*{TODO}
\subsection*{Sie kennen Beispiele von Angriffen auf verschiedenen Ebenen des Protokollstacks und wissen was diese bewirken}
\paragraph*{TODO}
\subsection*{Sie verstehen wie ein Cross-Site-Scripting/SQL-injection/Social-Engineering-Angriff abläuft und wie man sich dagegen schützen kann}
\paragraph*{TODO}

\part{Management (SW~07-09)}
\section{Standards \& Frameworks, ISMS}
\subsection*{Sie wissen, was ein ISMS ist und wie man damit umgeht}
\paragraph*{TODO}
\subsection*{Sie kennen die wichtigsten Standards der Informationssicherheit}
\paragraph*{TODO}
\subsection*{Sie finden sich in den Standards ISO 27001 und 27002 zurecht}
\paragraph*{TODO}
\subsection*{Sie verstehen die Grundzüge der BSI-Standards (BSI=Bundesamt für Sicherheit in der Informationstechnik, Deutschland)}
\paragraph*{TODO}
\subsection*{Sie kennen die Struktur und Grundziele des NIST CyberSecurityFrameworks}
\paragraph*{TODO}

\section{Risiko-Management und IT-Grundschutz}
\subsection*{Das Risikoanalyse-Verfahren verstehen}
\paragraph*{TODO}
\subsection*{Die Unterschiede zum Grundschutzverfahren kennen}
\paragraph*{TODO}
\subsection*{Eine einfache Risikoanalyse durchführen können}
\paragraph*{TODO}
\subsection*{Sie verstehen die Idee, die Ziele und die Konzepte des IT-Grundschutz-Vorgehens}
\paragraph*{TODO}
\subsection*{Sie kennen den Aufbau der IT-Grundschutz-Kataloge und deren Anwendungsweise}
\paragraph*{TODO}
\subsection*{Sie können die Teilschritte zum Aufbau eines Sicherheitskonzeptes nach IT-Grundschutz durchführen, kombinierte Risikoanalyse}
\paragraph*{TODO}

\section{Awarness}
\subsection*{Sie verstehen die Wichtigkeit der \flqq Awareness \frqq}
\paragraph*{TODO}
\subsection*{Sie kennen verschiedene Prozesse und Vorgehensweisenfür die Initiierung, Durchführung und Erfolgsprüfung einer Awareness-Kampagne und können diese anwenden}
\paragraph*{TODO}
\subsection*{Sie kennen die relevanten Erfolgsfaktorender Mitarbeiter-Sensibilisierung und -Schulung und können diese in einer Kampagne umsetzen}
\paragraph*{TODO}

\part{Access Control (SW~10)}
\section{Access Control}
\subsection*{Sie kennen verschiedene Arten der Authentisierung, wissen wie diese technisch ablaufen und was deren Vor- und Nachteile sind}
\paragraph*{TODO}
\subsection*{Sie wissen wie verschiedene Authentisierungstoken technisch funktionieren, was deren Vor- und Nachteile sind und wie sie beim Login oder bei der Transaktionsbestätigung im e-Banking eingesetzt werden}
\paragraph*{TODO}
\subsection*{Sie wissen was Authentisierung, Autorisierung ist, warum diese wichtig sind und wie Angriffe darauf ablaufen}
\paragraph*{TODO}

\part{Multi-Party-Computation (SW~11)}
\subsection*{Sie kennen einfache Beispiele von verteilten sicheren Berechnungen und verstehen wie die entsprechenden Protokolle ablaufen}
\paragraph*{TODO}
\subsection*{Sie kennen Arten von Sicherheit von verteilten sicheren Berechnungen und wie diese angegriffen werden können}
\paragraph*{TODO}
\subsection*{Sie wissen welche Eigenschaften elektronisches Geld ausmachen und kennen die technischen Grundlagen von Bitcoin}
\paragraph*{TODO}

\section{Cryptographic Protocols}
\section{Secret Sharing}
\section{Zero Knowledge Proof}
\subsection*{Sie wissen was Zero-Knowledge-Proofs sind und wie diese ablaufen}
\paragraph*{TODO}

\part{Quantum (SW~12)}
\subsection*{Sie wissen was ein Quantencomputer ist und was ihn von einem "`klassischen"' Computer unterscheidet}
\paragraph*{TODO}
\subsection*{Sie verstehen welchen Einfluss die Existenz eines Quantencomputers auf die Kryptographie hat}
\paragraph*{TODO}
\subsection*{Sie verstehen wie Quantenschlüsselaustausch funktionert}
\paragraph*{TODO}

\section{Quantum Computing and Quantum Cryptography}
\part{WAF, Federations (SW~13)}
\section{Firewalls}
\subsection*{Sie wissen was die Aufgaben einer Firewall sind}
\paragraph*{TODO}
\subsection*{Sie verstehen die Funktionsweise einer WAF und wie sie eine Webanwendung vor Angriffen schützen kann}
\paragraph*{TODO}

\section{Federations}
\subsection*{Sie verstehen wie Authentisierung mit Identity Federation abläuft, was die Voraussetzungen dafür sind und was die Vor- und Nachteile von Federations sind}
\paragraph*{TODO}

\part{Talks (SW~14)}
\section{Malware}
\subsection*{Sie verstehen, welche Arten von Malware es gibt, welche Massnahmen gegen Malware sinnvoll sind und wie diese wirken}
\paragraph*{TODO}

\section{WAF}
\subsection*{Sie verstehen wo Machine-Learning in einer WAF eingesetzt werden kann und was einene Machine-Learning-Ansatz vom "`herkömmlichen"' Einsatz einer WAF unterscheidet}
\paragraph*{TODO}
\subsection*{Sie kennen Beispiele von Angriffen, welche mittels Machine-Learning auf einer WAF erkannt werden konnten}
\paragraph*{TODO}
\end{document}