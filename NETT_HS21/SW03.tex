\part{SW 03 - Präsentationen zu physikalischer Schicht}
\section{Lernziele (Leitfragen)}
\begin{itemize}
    \item Die physikalische Schicht und Zugriffsverfahren (T1)
    \begin{enumerate}
        \item Was ist der Zweck der physikalischen Schicht?
        \item Was sind die Hauptmerkmale der physikalischen Schicht?
        \item Was ist der Unterschied zwischen \flqq{}Simplex\frqq, \flqq{}half-duplex\frqq{} and \flqq{}full duplex\frqq?
        \item Welches sind die am häufigsten verwendeten Zugriffsverfahren?
        \item Was bedeutet „Late Collision“?
        \item Was muss man noch unbedingt über die physikalische Schicht und Zugriffsverfahren wissen?\\
    \end{enumerate}

    \item Topologien und ``Bandwidth'' (T2)
    \begin{enumerate}
        \item Was für Topologien findet man in Computernetzwerken?
        \item Wo ist der Unterschied zwischen \flqq{}Bandwidth\frqq, \flqq{}Throughput\frqq{} und \flqq{}Goodput\frqq? Wie kann man diese Konzepte visualisieren und verstehen?
        \item Was ist \flqq{}Latency\frqq{} und \flqq{}Jitter\frqq? Wie kann man diese Konzepte visualisieren und verstehen?
        \item Was muss man noch unbedingt über Topologien und ``Bandwidth'' wissen?\\
    \end{enumerate}

    \item Kupferkabel (T3)
    \begin{enumerate}
        \item Was sind die wichtigsten Merkmale von Kupferkabeln?
        \item Was für Kupferkabelarten werden heutzutage in Computernetzwerken am häufigsten verwendet?
        \begin{enumerate}
            \item Wie sind sie aufgebaut?
            \item Wie sehen die Stecker aus?
        \end{enumerate}
        \item Worauf muss bei der Handhabung und Verlegung der Kupferkabel besonders geachtet werden und warum?
        \item Woraus resultieren die Längenbeschränkungen der Kupferverkabelung?
        \item Was muss man noch unbedingt über Kupferkabel wissen?\\
    \end{enumerate}

    \item Glasfaserkabel (T4)
    \begin{enumerate}
        \item Was sind die wichtigsten Merkmale von Glasfaserkabeln?
        \begin{enumerate}
            \item Wie sind sie aufgebaut?
            \item Wie sehen die Stecker aus?
        \end{enumerate}
        \item Worauf muss bei der Handhabung und Verlegung von Glasfaserkabeln besonders geachtet werden und warum?
        \item Woraus resultieren die Längenbeschränkungen der Glasfaserkabelverkabelung?
        \item Wo ist der Unterschied zwischen Multi- und Singlemode (Monomode)- Glasfasern?
        \item Was sind die Vor- und Nachteile von Glasfaserkabel (im Vergleich zu Kupferkabeln)?
        \item Was muss man noch unbedingt über Glasfaserkabel wissen?\\
    \end{enumerate}

    \item Wireless Access (T5)
    \begin{enumerate}
        \item Was sind die wichtigsten Merkmale von \flqq{}Wireless Media\frqq?
        \item Welche Wireless Access Geräte arbeiten auf Layer I?
        \item Was für Wireless Standards gibt’s in Computernetzwerken?
        \begin{enumerate}
            \item Was sind ihre Hauptmerkmale und Anwendungsbereiche?
        \end{enumerate}
        \item Was sind die Vor- und Nachteile von \flqq{}Wireless Access\frqq{} Methoden im Vergleich mit \flqq{}Wired Access\frqq?
    \end{enumerate}
\end{itemize}

\section{Antworten T1}
\subsection*{Was ist der Zweck der physikalischen Schicht?}
//TODO
\subsection*{Was sind die Hauptmerkmale der physikalischen Schicht?}
//TODO
\subsection*{Was ist der Unterschied zwischen \flqq{}Simplex\frqq, \flqq{}half-duplex\frqq{} and \flqq{}full duplex\frqq?}
//TODO
\subsection*{Welches sind die am häufigsten verwendeten Zugriffsverfahren?}
//TODO
\subsection*{Was bedeutet „Late Collision“?}
//TODO
\subsection*{Was muss man noch unbedingt über die physikalische Schicht und Zugriffsverfahren wissen?}
//TODO

\section{Antworten T2}
\subsection*{Was für Topologien findet man in Computernetzwerken?}
//TODO
\subsection*{Wo ist der Unterschied zwischen \flqq{}Bandwidth\frqq, \flqq{}Throughput\frqq{} und \flqq{}Goodput\frqq? Wie kann man diese Konzepte visualisieren und verstehen?}
//TODO
\subsection*{Was ist \flqq{}Latency\frqq{} und \flqq{}Jitter\frqq? Wie kann man diese Konzepte visualisieren und verstehen?}
//TODO
\subsection*{Was muss man noch unbedingt über Topologien und ``Bandwidth'' wissen?}
//TODO

\section{Antworten T3}
\subsection*{Was sind die wichtigsten Merkmale von Kupferkabeln?}
//TODO
\subsection*{Was für Kupferkabelarten werden heutzutage in Computernetzwerken am häufigsten verwendet?}
//TODO
\subsubsection*{Wie sind sie aufgebaut?}
//TODO
\subsubsection*{Wie sehen die Stecker aus?}
//TODO
\subsection*{Worauf muss bei der Handhabung und Verlegung der Kupferkabel besonders geachtet werden und warum?}
//TODO
\subsection*{Woraus resultieren die Längenbeschränkungen der Kupferverkabelung?}
//TODO
\subsection*{Was muss man noch unbedingt über Kupferkabel wissen?}
//TODO

\section{Antworten T4}
\subsection*{Was sind die wichtigsten Merkmale von Glasfaserkabeln?}
//TODO
\subsubsection*{Wie sind sie aufgebaut?}
//TODO
\subsubsection*{Wie sehen die Stecker aus?}
//TODO
\subsection*{Worauf muss bei der Handhabung und Verlegung von Glasfaserkabeln besonders geachtet werden und warum?}
//TODO
\subsection*{Woraus resultieren die Längenbeschränkungen der Glasfaserkabelverkabelung?}
//TODO
\subsection*{Wo ist der Unterschied zwischen Multi- und Singlemode (Monomode)- Glasfasern?}
//TODO
\subsection*{Was sind die Vor- und Nachteile von Glasfaserkabel (im Vergleich zu Kupferkabeln)?}
//TODO
\subsection*{Was muss man noch unbedingt über Glasfaserkabel wissen?}
//TODO

\section{Antworten T5}
\subsection*{Was sind die wichtigsten Merkmale von \flqq{}Wireless Media\frqq?}
//TODO
\subsection*{Welche Wireless Access Geräte arbeiten auf Layer I?}
//TODO
\subsection*{Was für Wireless Standards gibt’s in Computernetzwerken?}
//TODO
\subsubsection*{Was sind ihre Hauptmerkmale und Anwendungsbereiche?}
//TODO
\subsection*{Was sind die Vor- und Nachteile von \flqq{}Wireless Access\frqq{} Methoden im Vergleich mit \flqq{}Wired Access\frqq?}
//TODO