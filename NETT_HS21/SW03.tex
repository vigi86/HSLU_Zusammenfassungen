\part{SW 03 - Präsentationen zu physikalischer Schicht}
\section{Lernziele (Leitfragen)}
\begin{itemize}
    \item Die physikalische Schicht und Zugriffsverfahren (T1)
    \begin{enumerate}
        \item Was ist der Zweck der physikalischen Schicht?
        \item Was sind die Hauptmerkmale der physikalischen Schicht?
        \item Was ist der Unterschied zwischen «Simplex», «half-duplex» and «full duplex»?
        \item Welches sind die am häufigsten verwendeten Zugriffsverfahren?
        \item Was bedeutet „Late Collision“?
        \item Was muss man noch unbedingt über die physikalische Schicht und Zugriffsverfahren wissen?
    \end{enumerate}

\end{itemize}

\section{Antworten}
\subsection*{Was ist der Zweck der physikalischen Schicht?}
\subsection*{Was sind die Hauptmerkmale der physikalischen Schicht?}
\subsection*{Was ist der Unterschied zwischen «Simplex», «half-duplex» and «full duplex»?}
\subsection*{Welches sind die am häufigsten verwendeten Zugriffsverfahren?}
\subsection*{Was bedeutet „Late Collision“?}
\subsection*{Was muss man noch unbedingt über die physikalische Schicht und Zugriffsverfahren wissen?}