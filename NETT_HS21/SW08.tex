\part{SW 08}
\section{Lernziele (Leitfragen)}
\begin{itemize}
    \item
    \item
    \item
    \item
    \item
    \item
    \item
    
    SW 08 & 09 – T1

Wie wird eine DNS Anfrage bearbeitet?
•	 
Abbildung 15: Grafik aus Unterricht
•	Forward Lookup
o	Ich kenne die IP noch nicht
o	Steps
1.	Client (DNS Client, sucht etwas) 
•	Client muss wissen, welchen Server er kontaktieren muss
•	Client fragt nach www.yahoo.com
2.	ISP (Internet Service Provider) 
•	Erhält die Anfrage des Clients
•	Kennt die www.yahoo.com noch nicht
•	Der ISP geht zu einem der Root Server
3.	Root Server
•	Erhält die Anfrage des ISP und sagt, frag den .com Server
4.	Alles weiter bis man bei yahoo.com ist
•	Dieser meldet sich beim ISP und der ISP meldet an den Client, die IP des Servers von yahoo.com
•	Reverse Lookup 
o	Ich kenne den Host Name aber die IP noch nicht
•	DNS Cache dient dazu, dass nicht jedesmal das ganze Spiel gemacht werden muss, werden die Angaben zwischengespeichert
 
Was ist der Unterschied zwischen rekursiver und iterativer DNS?
•	Rekursive Query
o	 
Abbildung 16: Screenshot YT https://www.youtube.com/watch?v=PS0UppB3-fg
•	Iterative Query
o	 
Abbildung 17: Screenshot YT https://www.youtube.com/watch?v=PS0UppB3-fg

Was für DNS Record Arten gibt es?
•	CNAME
•	A
•	AAAA
•	MX Record
•	TXT
•	SRV

Was ist die Topologie des DNS Systems? Ist es zentralisiert?
•	Es ist verteilt, verschiedene Server haben Zugriff
 

Wie kann ich direkt eine DNS Anfrage aus meinem Computer ausführen?
•	Mittels NS Lookup kann man die IP oder Domain eines bestimmten Computers herausfinden

Was sind die Sicherheitsmerkmale von DNS?
•	Kennt nur System innerhalb des Netzwerks
•	Initial wurden keine Sicherheitsmechanismen eingebaut. Diese wird gewährleistet durch die Anbieter, z. B. von CloudFlare. Diese bieten Schutz vor DDoS-Attacken, DNS-Spoofing etc. an. 

Was für Sicherheitserweiterungen gibt es für DNS?
o	Domain Name System Security Extensions (DNSSEC) DNSSEC verhindert, dass Angreifer die Antworten auf DNS-Anfragen verfälschen oder manipulieren. 


Wie geht DNS mit den verschiedenen IP Versionen um?
•	A für ipv4
•	AAAA für ipv6

SW 08 & 09 – T2

Wie erhält ein Host seine IPv4 Konfiguration mit DHCP?
•	DHCP oder «Dynamic Host Configuration Protocol» wird in der Netzwerktechnik verwendet, um einem Client alle nötigen Netzwerkinformationen zuzuweisen, wie: 
o	IP-Adresse 
o	Subnetzmaske 
o	Standardgateway 
o	Domain Name Server (DNS)
•	Es können mehrere DHCP Server in einem Netzwerk vorhanden sein, dies ist für die Availability des Services von Vorteil, sollte ein DHCP Server Probleme aufweisen
•	Discover-Offer-Request-Acknowledgement (DORA)
o	DHCP Discover als Broadcast
o	DHCP Offer
	Mit Client MAC, IP, Subnetz, Gateway, Leasttime und IP des DHCP
o	DHCP Request als Broadcast
	Annahme der IP Daten 
o	DHCP Acknowledgment
	Bestätigung mit weiteren Optionen


Welche Nachrichten des DHCP Protokolls sind Broadcasts und wieso?
•	Discover, weil er das Netzwerk noch nicht kennt und die IP des DHCP Servers braucht
•	Request, weil alle im Netz wissen müssen, dass diese IP nun vergeben wurde
 

Was passiert, wenn mehr als ein DHCP Server in dem lokalen Netzwerk verfügbar ist? Ist das möglich? Ist das wünschenswert? (noch leer!)
•	Hat man mehrere DHCP Server im Netz, gibt es mehrere Offers
•	Mittels Request der als Broadcast fungiert werden alle DHCP Servers darüber informiert, dass der Client diese gewählt hat

Welche Parameter werden typischerweise von einem DHCP Server vergeben?
•	IP, Subnetz, Standardgateway

Was macht ein Host, wenn er keine IPv4 Konfiguration via DHCP bekommt? Kann er mit anderen Hosts kommunizieren?
•	Werden keine statischen IP-Adressen an einen Client vergeben, weisen sich die Clients bei einem Ausfall vom DHCPO automatisch eine IP-Adresse in folgenden IP-Range zu: 169.254.0.0- 169.254.255.255. 
•	Diese sogenannten Link-Local Adressen ermöglichen eine Kommunikation in einem gemeinsamen lokalen Netzwerk

Wie erhält ein Host seine IPv6 Konfiguration mit DHCPv6?
•	 Für DHCPv6 gibt es zwei verschiedene Verfahren: 
•	Stateless Config: Router verteilt die IPv6 Präfix und der DHCPv6 die restlichen Parameter 
•	Statefull Config: Der DHCPv6 verteilt IPv6 Präfix als auch die restlichen Parameter Für IPv6 wird das Protokoll DHCPv6 benützt

SW 08 & 09 – T3

Wie wird ein E-Mail mit SMTP verschickt (end-to-end)?
•	SMTP gehört zur Anwendungsschicht, bedeutet Simple Mail Transfer Protocol und dient, wie der Name sagt, zum Austauschen von E-Mails. 
•	Simple Mail Transfer Protcol
•	Kann auch als Gedanken stütze mit Sending-Mail-To-People genutzt werden
•	Schritte
1.	Ein Mail-Client (Outlook) sendet E-Mail an SMTP Server.
	Wenn Gmail genutzt wird, ist dieser Server smtp.gmail.com
2.	Dieser SMTP-Server sendet dann die Mail an den SMTP Server des Empfängers. 
3.	E-Mail wird vom SMTP Server des Empfängers erhalten
4.	Hier endet das SMTP. Um die E-Mails abzurufen, kommen dann IMAP/POP3 zum tragen 
•	SMTP funktioniert über TCP
 
Wie wird der Nutzer des SMTP Servers authentifiziert?
•	Authentifizierung: Beispiel mit Brief Absenderadresse, die vom Versender angepasst werden kann. Dies soll verhindert werden mit SMTP. 
•	SMTP Auth ist Extension von Extended SMTP was wiederum eine Erweiterung von SMTP ist.
•	Somit können nur noch Vertrauenswürdige Nutzer Emails über diesen Sender Server versenden. 
•	Statt über den Standardport 25/TCP wird über den Port 587 kommuniziert. Obligatorische Grundlage für E SMTP. Verschiedene Authentifizierungsmechanismen (PLAIN, LOGIN, CRAM-MD5).
•	 
Abbildung 18: SMTP Eigene Grafik


Welche Sicherheitseigenschaften hat SMTP? Was für Sicherheitserweiterungen gibt es?
•	Eine Vertraulichkeit, Authentizität und Integrität von E-Mails kann durch SMTP allein nicht gewährleistet werden. Clientseitige Verfahren müssen genutzt werden. 
•	Ansonsten sind die Absender und Empfänger fälschbar und die E-Mailinhalte grundsätzlich lesbar und veränderbar. 
•	S/MIME ist eine Technologie, die E-Mails verschlüsselt, um sie vor unerwünschtem Zugriff zu schützen. Ausserdem können die E-Mails digital signiert werden, um den Absender als legitim zu verifizieren. 
•	PGP (Pretty Good Privacy) ist eine alternative Methode E-Mails zu signieren und verschlüsseln. Anders als S/MIME, wo eine Root Certificate Authority Zertifikate ausstellt, basiert PGP auf ein Web of Trust. 
•	TLS: Verschlüsselt die Verbindung und Daten während der Übertragung von Punkt A nach Punkt B.
 
Kriegt die Empfängerin das E-Mail sobald es von dem SMTP Server empfangen wird?
•	Wenn ein Server eine Nachricht erhält, legt er die Nachricht entweder lokal ab oder leitet sie einem anderen E-Mail-Server weiter. 
•	Ist der Destination E-Mail-Server nicht online, oder beschäftigt, so sendet SMTP die Nachricht zu einem späteren Zeitpunkt. Wenn sie nach einer bestimmten Zeit immer noch nicht zugestellt werden kann, dann wird sie dem Sender als unzustellbar zurückgeschickt. 


Wie wird auf E-Mails mit POP3 zugegriffen?
•	POP3 ist die dritte Version des Post Office Protocols. Es ist das Übertragungsprotokoll für E-Mails und stammt aus dem Jahr 1996.
•	Dabei verbindet sich der POP3 Client mit dem Mailserver und authentifiziert sich durch ein Passwort. 
•	Sodann ruft der Client neue Nachrichten für die Mailadresse ab und der Server sendet diese E-Mails an den Client. 
•	Nach der Übertragung löscht der Server die Nachrichten.

Wie wird auf E-Mails mit IMAP zugegriffen?
•	IMAP steht für Internet Message Access Protocol. 
•	Bei IMAP basierten Mailclients werden die Mails sowie die Ordnerstrukturen und Einstellungen auf einem Mailserver gespeichert. 
•	Der Client holt sich dann die einzelnen Informationen erst vom Server, wenn sie gebraucht werden. 
•	Eine normale IMAP Kommunikation beginnt mit einem Login, wobei sich der Client mit einem Username und einem Password beim Mailserver authentifiziert. 
•	Danach kann der Client dem Server diverse Anfragen stellen, um Infos über die Mails zu bekommen, oder um die Mails auf dem Server zu bearbeiten (verschieben, löschen, markieren, etc.)

Was sind die Hauptunterschiede zwischen POP3 und IMAP? Wann wird es empfohlen sie zu verwenden?
•	Beim POP3 werden die Daten lokal abgespeichert und auf dem Server gelöscht. Mit dem IMAP ruft der Client die Daten vom Server ab. 
•	Möchte man wenig Bandbreite nutzen, sollte man POP3 verwenden. 
•	Will man mit mehreren Devices auf die Mails zugreifen und ein sicheres Backup haben, empfiehlt es sich das IMAP zu verwenden.
SW 08 & 09 – T4 

Wie wird auf eine Website mit HTTP(S) zugegriffen?
•	Nach der DNS-Auflösung wird mit der GET Methode von HTTP(s) auf Webseite über den Port 80 (443 bei HTTPS) zugegriffen. 
•	Wird diese GET Anfrage vom Webserver akzeptiert, so sendet er erst Inhalt des Headers des Bereitgestellten HTML Dokuments und im Anschluss den Body. Der Body repräsentiert den Inhalt einer Webseite, und ist das, was der User im Interface seines Webbrowsers sieht. 
•	Im Anschluss wird die Verbindung beendet
 
Was ist der Unterschied zwischen HTTP und HTTPS?
•	Das “S” von HTTPS steht für Secure. 
•	Bei einer HTTPS Verbindung wird das Webbrowser Zertifikat zum Verschlüsseln der Verbindung verwendet. 
•	Somit ist die Verbindung über SSL/TLS (Transport Layer Security) gesichert und die gesendeten Daten, können nicht oder nur schwer abgehört werden. 
•	HTTPS ist heutzutage beinahe ein Standard für alle offiziellen Webseiten

Was sind die Hauptmerkmale von TLS? Wie funktioniert TLS (hohes Niveau)?
•	TLS (Transport Layer Security) ist ein Protokoll der 5 Schicht, welches zuständig ist für eine sichere Datenübertragung im Internet. 
•	Nebst dem HTTPS Protokoll können auch Protokolle wie SMTP, FTP, POP3, etc. das TLS Protokoll verwenden. 
•	Funktionsweise: 
•	Die Funktion ist in zwei Phasen unterteilt. 
o	Die erste Phase ist der Verbindungsaufbau und die zweite Phase ist die Übermittlung.
•	Es wird mit zwei verschiedenen Schlüsseln gearbeitet (Private und Public Key). 
•	Die Public Key vom Sender ist jeweils dem Empfänger bekannt. 
•	Mit dem Public Key werden Daten verschlüsselt und mit dem Private Key werden die Daten entschlüsselt. 
•	Bevor die Verschlüsselung startet, wird überprüft, ob der Empfänger den echten öffentlichen Schlüssel mitteilt. Die Überprüfung findet mit Hilfe von den Zertifikaten statt

Was sind andere wichtige Verwendungen von HTTP (ausser Websites zuzugreifen)?
•	Rest API→ Schnittstellen bzw. Webservices verwenden HTTP

Was ist ein REST API?
•	REST = Representational State Transfer API = 
•	API = Application Programming Interface => Programmierschnittstelle, die den Austausch von Informationen ermöglicht, wenn diese sich auf unterschiedlichen Systemen befinden. Verwendet HTTP-Anfragen.

Was sind die Hauptmethoden von HTTP?
•	GET, POST, PUT, DELETE, HEAD, CONNECT, OPTIONS, TRACE

Wie funktioniert ein REST API?
•	Die REST Schnittstelle nutzt HTTP-Anfragen, um mit GET, POST, PUT und DELETE auf Informationen zuzugreifen. 
•	Jede URL wird als Anforderung bezeichnet, während die zurückgegeben Daten die Antwort sind. Sobald eine Client-Anfrage auf dem Server eingegangen ist, sucht die REST-API nach einer Antwort und liefert sie unverzüglich.
 
Nenne ein Beispiel von einem REST API Endpoint, der die GET Methode verwendet.
•	Aufrufen einer Ressource --> Bsp: Facebook Profil aufrufen

Nenne ein Beispiel von einem REST API Endpoint, der die POST Methode verwendet.
•	Anlegen einer Ressource --> Bsp: Facebook Account erstellen

Nenne ein Beispiel von einem REST API Endpoint, der die PUT Methode verwendet.
•	Verändern einer Ressource --> Bsp: Facebook status ändern

SW 08 & 09 – T5 

Wofür wird SSH verwendet?
•	Secure Shell (SSH) bezeichnet ein Protokoll, in welchem Clients auf entfernte Computer zugreifen können. 
•	Administratoren können damit einen Computer durch Fernzugriff konfigurieren und betreuen.
•	Es gibt kein GUI dazu (hosttest.de, 2022)

Wie funktioniert SSH (hohes Niveau)?
•	SSH gibt’s nur auf Linux, auf Windows ist es PowerShell eignet sich aber nur beschränkt für Fernzugriff
•	Verbindung mit SSh benötigt IMMER zwei Programme
o	Server wie z.B. OpenSSH Server auf entferntem Computer
o	Client wie SSH (Linux) oder PuTTY (Windows) auf lokalem Rechner
1.	Verbindungsaufbau zum Server via Hostname, Domain oder IP 
2.	Wird Anfrage entgegengenommen muss Nutzer mit Namen & Passwort oder durch digitales Zertifikat identifizieren
3.	Dann steht textbasierte Umgebung (Shell) auf Server zur Verfügung und es kann gearbeitet werden (hosttest.de, 2022)

Was für Nutzerauthentifizierungsoptionen gibt es? Was wir empfohlen?
•	SSH
•	2-Factor-Authentication
o	Passwort und E-Mail/SMS Code
o	Doppelte Sicherheit
•	External Key’s
•	OAuth (varonis.com, 2022)
o	Open Standard Authorization protocol
o	Authentifizierung dass beispielsweise ESPN.com auf meine Facebook Profil Infos zugreifen darf, ohne mein Passwort zu kennen
o	Benutzt JSON
o	Wird für moderne Applikationen mittels API calls genutzt
•	SAML (varonis.com, 2022) Securtiy Assertion Markup Language
o	SSO Single Sign on
o	SAML benutzt XML
o	SAML arbeitet mit Session cookie -> gut für Arbeitstage aber nicht, wenn ich jedesmal auf meine IoT Lampe/Backofen etc. zugreifen will
 
Was ist RDP?
•	RDP Remote Desktop Protocol
•	Damit kann auf einen entfernten Desktop zugegriffen werden.

Wie funktioniert RDP?
•	User kann Aktionen auf dem entfernten Computer durchführen
•	RDP öffnet einen dedizierten Kanal zwischen zwei Verbundenen Geräten
•	Es nutzt immer Port 3389
•	Via TCP/IP werden die Kommandos ausgetausch
•	RDP verschlüsselt alle Daten damit die Verbindung noch sicherer ist
•	(cloudflare.com, 2022)

Was ist VNC?
•	VNC Virtual Network Computing
•	Wird im übergeordneten als Remote Desktop Sharing bezeichnet
•	User können damit den Computer aus dem Geschäft zuhause anzeigen lassen
•	(vpnchecked.com, 2022)

Wie funktioniert VNC?
•	Funktioniert im Client-Server Modell
•	User muss nur einen VNC Viewer auf einem lokalen Computer (client) haben. Dieser verbindet sich von entfernt auf einen anderen Computer, wo VNC als Server installiert ist. 
•	VNC ist Plattformunabhängig, beide Computer müssen lediglich TCP/IP aktiviert haben und offene Ports mit zugelassenem Traffic
•	 
Abbildung 19: VNC, Screenshot von vpnchecked.com
•	(vpnchecked.com, 2022)

Was ist der Unterschied zwischen RDP und VNC? 
•	Grundsätzlich: zwei verschiedene Protokolle
o	RDP: «Client zu Client»
o	VNC: «Client zu Server»
•	RDP ist schneller und eignet sich für Virtualisierung besser
•	RDP unterstützt SSL/TLS und bekommt Security Updates
•	Nicht alle VNC Softwares akzeptieren SSH, VNC gibt den Clients «Full Access»
•	(differencebetween.net, 2022) 
Was ist VDI?
•	VDI Virtual Desktop Infrastructure
•	Sorgt dafür, dass ein Geschäftscomputer von überall her zugreifbar ist

Wie funktioniert VDI?
•	Komplexer als einfache RDP weil noch Server etc. virtualisiert drinhängen
•	Desktop OS ist meisten in einem Zentralisierten Server oder einem physischen Datencenter gehostet.
•	Zwei Arten
o	Persistent Virtual Desktop
o	Speichern für Zukünftige Nutzung, traditioneller Desktop
o	NonPersistent
o	Einheitliche Desktops, wo man auf das zugreifen kann, was man braucht
o	Desktop geht zurück in Einheitlichen Status nachdem der User sich ausloggt
•	(azure.microsoft.com, 2022)

Nenne Beispiele von kommerziellen VDI Lösungen
•	Citrix Workspace
•	VirtualBox
•	VM Fusion
•	Amazon WorkSpaces

    
    
    \item
    \item
    \item
    \item
    \item
    \item
    \item
    \item
\end{itemize}

\section{Antworten}
\subsection*{}
